\input{woodamble}                                                       % Custom preamble
%----------------------------------------------------------------------------------------
% NAME AND CLASS SECTION
%----------------------------------------------------------------------------------------

\newcommand{\reportTitle}{Practical 7 Report} % Assignment title
\newcommand{\reportDueDate}{Wednesday,\ September\ 03,\ 2014}                    % Due date
\newcommand{\reportClass}{EEE3017W - Digitals}                               % Course/class
\newcommand{\reportClassTime}{10:30am}                                 % Class/lecture time
\newcommand{\reportClassInstructor}{Jones}                               % Teacher/lecturer
\newcommand{\reportAuthorName}{Sean Wood - WDXSEA003 $|$ Sean Woodgate - WDGSEA001}                           % Your name
\newcommand{\reportDepartment}{Department of Electrical Engineering}           % Department

%----------------------------------------------------------------------------------------
% TITLE PAGE
%----------------------------------------------------------------------------------------

\title{
\begin{figure}[H]
  \begin{center}
    \includegraphics[width=0.4\columnwidth]{uct-logo}
  \end{center}
\end{figure}
\textmd{\Huge UNIVERSITY OF CAPETOWN \\ \LARGE \reportDepartment} \\
\vspace{2in}
\textmd{\textbf{\LARGE \reportClass\\ \Huge \reportTitle \\ \vspace{1.5in} \Large \reportAuthorName}}\\
%\normalsize\vspace{0.1in}\small{Due\ on\ \hmwkDueDate}\\
%\vspace{0.1in}\large{\textit{\hmwkClassInstructor\ \hmwkClassTime}}
}

\date{}

%========================================================================================
%========================================================================================

\begin{document}
  \maketitle
  \thispagestyle{empty}

% \frontmatter
%----------------------------------------------------------------------------------------
% ABTRACT
%----------------------------------------------------------------------------------------
\setcounter{tocdepth}{3}                                                      % ToC Depth

% \begin{abstract}
% This report covers the design and testing of a state feedback controller implemented to control a computer model of a helicopter.  The controller interfaces with the computer by means of a ADC and DAC and is comprised of operational amplifiers. % TODO put more here!s
% \end{abstract}
% \newpage

%----------------------------------------------------------------------------------------
% TABLE OF CONTENTS
%----------------------------------------------------------------------------------------

% \clearpage
%
% \tableofcontents
%
% \clearpage
%
% \listoffigures
% \listoftables
%
% \clearpage
%
%----------------------------------------------------------------------------------------
% NOMENCLATURE
%----------------------------------------------------------------------------------------

% \section{Nomenclature}
% \label{sec:Nomenclature}
%
% \textbf{Constants}
%
% \(V_{LINE}\) - Nominal Line Voltage\\
% \(A_v\) - Operational Amplifier Gain\\
% \(db_{SPL}\) - Sound Pressure Level\\
% \(\theta_S\) - Phase Shift

%----------------------------------------------------------------------------------------
% INTRODUCTION
%----------------------------------------------------------------------------------------
\section{Introduction}
\label{sec:Introduction}
This report covers the system design and implementation of a simple stopwatch which has lap functionality.

\section*{System Design}
\label{sec:System Design}
\subsection*{(a)}
\label{sub:(a)}
  Below are the calculations used to find the timer parameters to result in a 0.01s period. The calculations do not follow from the practical instructions.

  \begin{equation}
    \begin{split}
      f_{cnt} &= \frac{f_{clk}}{\text{Prescaler}}\\
      \text{Prescaler} &= 48\\
      \therefore f_{cnt} &= \frac{48000000}{48} = 1000000 \text{Hz}\\
      \text{ARR} &= T_{tim} \times f_{cnt}\\
      &= 0.01 1000000\\
      &= 10000 \eoc
    \end{split}
  \end{equation}

\newpage
\subsection*{(b)}
\label{sub:(b)}
  Below is the initialization function for the ports used in the project: \\
  \begin{minted}{c}
    /*
     * @brief Initialise the GPIO ports for pushbuttons, LEDs and the ADC
     * @params None
     * @retval None
     */
    static void init_ports(void) {
      // Enable the clock for ports used
      RCC->AHBENR |= RCC_AHBENR_GPIOBEN | RCC_AHBENR_GPIOAEN;

      // Initialise PB0 - PB7, PB10 and PB11 for RG Led
      GPIOB->MODER |= GPIO_MODER_MODER0_0 | GPIO_MODER_MODER1_0 |
                      GPIO_MODER_MODER2_0 | GPIO_MODER_MODER3_0 |
                      GPIO_MODER_MODER4_0 | GPIO_MODER_MODER5_0 |
                      GPIO_MODER_MODER6_0 | GPIO_MODER_MODER7_0 |
                      GPIO_MODER_MODER10_0 | GPIO_MODER_MODER11_0;
      GPIOB->ODR &= ~(GPIO_ODR_10 | GPIO_ODR_11); // Make sure they are not on

      // Initialise PA0, PA1, PA2 and PA3 for SW0, SW1, SW2 and SW3
      GPIOA->MODER &= ~(GPIO_MODER_MODER0 | GPIO_MODER_MODER1 | GPIO_MODER_MODER2
          | GPIO_MODER_MODER3);
      GPIOA->PUPDR |= GPIO_PUPDR_PUPDR0_0 | GPIO_PUPDR_PUPDR1_0
          | GPIO_PUPDR_PUPDR2_0 | GPIO_PUPDR_PUPDR3_0; // Enable pullup resistors

      // Initialise PA5 for ADC1
      GPIOA->MODER |= GPIO_MODER_MODER5;
    }
  \end{minted}
\vspace{0.5cm}

Below is the initialization function for TIM14:\\

\begin{minted}{c}
  /*
   * @brief Initialise the TIM14
   * @params None
   * @retval None
   */
  static void init_TIM14(void) {
    // Enable the clock for TIM14
    RCC->APB1ENR |= RCC_APB1ENR_TIM14EN;

    // Set the frequency to 100Hz
    TIM14->PSC = 48;
    TIM14->ARR = 10000;

    // Enable the interrupt
    TIM14->DIER |= 0x1; // Enable the UIE (Update Interrupt Enable)
    TIM14->CR1 &= ~(1 << 2); // Make sure the interrupt is not disabled in the Control Register 1

    // Make sure the counter is at zero
    TIM14->CNT = 0;
  }
\end{minted}
\vspace{0.5cm}

Below is the initialization function for the Nested Vector Interrupt Controller (NVIC):\\

\begin{minted}{c}
  /*
   * @brief Initialise the NVIC for pushbutton interrupts
   * @params None
   * @retval None
   */
  static void init_NVIC(void) {
    NVIC_EnableIRQ(EXTI0_1_IRQn); // For lines 0 and 1
    NVIC_EnableIRQ(EXTI2_3_IRQn); // For lines 2 and 3
    NVIC_EnableIRQ(TIM14_IRQn); // For TIM14
  }
\end{minted}
\vspace{0.5cm}

\newpage
\subsection*{(c)}
\label{sub:(c)}
  Below is the TIM14 interrupt handler which simply increments a global variable which is keeping the time:
  \begin{minted}{c}
    /*
     * @brief Interrupt Request Handler for TIM14
     * @params None
     * @retval None
     */
    void TIM14_IRQHandler(void) {
      if (programState != PROG_STATE_STOP) {
        // If we are counting either in LAP mode or in COUNTING mode, increment the time
        timer++;
        if (programState == PROG_STATE_COUNTING) {
          // If we are in COUNTING mode, display the timer on the screen
          display(TIME, timer);
        }
      }

      // Clear the interrupt pending bit
      TIM14->SR &= ~(1 << 0);
    }
  \end{minted}

\newpage
\subsection*{(d)}
\label{sub:(d)}

  Below is the function to check for buttons. This function is used in conjunction with interrupts on the EXTI lines:\\

  \begin{minted}{c}
    /*
     * @brief Get the state of the specified switch, with debouncing of predefined length
     * @params pb: Pushbutton number
     * @retval True or false when pressed and not pressed rsp.
     */
    static uint8_t check_pb(uint8_t pb) {
      uint8_t pbBit;

      // Check which PB needs to be checked
      switch (pb) {
      case 0:
        pbBit = GPIO_IDR_0;
        break;
      case 1:
        pbBit = GPIO_IDR_1;
        break;
      case 2:
        pbBit = GPIO_IDR_2;
        break;
      case 3:
        pbBit = GPIO_IDR_3;
        break;
      default:
        return FALSE;
      }

      // Debounce and check again - return the result
      if (!(GPIOA->IDR & pbBit)) {
        delay(DEBOUNCE_MS * 1000);
        if (!(GPIOA->IDR & pbBit)) {
          return TRUE;
        } else {
          return FALSE;
        }
      } else {
        return FALSE;
      }
    }
  \end{minted}
  \vspace{0.5cm}

  \begin{minted}{c}
    /*
     * @brief Interrupt Request Handler for EXTI Lines 2 and 3 (PB0 and PB1)
     * @params None
     * @retval None
     */
    void EXTI0_1_IRQHandler(void) {
      if (check_pb(0)) {
        if (programState == PROG_STATE_STOP || programState == PROG_STATE_LAP) {
          // Put the program into COUNTING mode and set the appropriate LED
          display(TIME, timer);
          programState = PROG_STATE_COUNTING;
          GPIOB->ODR = (1 << 0);
        }
      } else if (check_pb(1)) {
        if (programState == PROG_STATE_COUNTING) {
          // Update program state to LAP mode
          programState = PROG_STATE_LAP;

          // Capture the lap time, display on the LCD and set the appropriate LED
          lapValue = timer;
          display(TIME, timer);
          GPIOB->ODR = (1 << 1);
        }
      }

      // Clear the interrupt pending bit
      EXTI->PR |= EXTI_PR_PR0 | EXTI_PR_PR1;
    }

    /*
     * @brief Interrupt Request Handler for EXTI Lines 2 and 3 (PB2 and PB3)
     * @params None
     * @retval None
     */
    void EXTI2_3_IRQHandler(void) {
      if (check_pb(2)) {
        if (programState == PROG_STATE_COUNTING) {
          // Put the program into STOP mode and set the appropriate LED
          programState = PROG_STATE_STOP;
          GPIOB->ODR = (1 << 2);
        }
      } else if (check_pb(3)) {
        // Zero the timer, update the program state, display the welcome screen and set the appropriate LED
        timer = 0;
        programState = PROG_STATE_STOP;
        display(WELCOME, 0);
        GPIOB->ODR = (1 << 3);
      }

      // Clear the interrupt pending bit
      EXTI->PR |= EXTI_PR_PR2 | EXTI_PR_PR3;
    }
  \end{minted}
  \vspace{0.5cm}

\newpage
\subsection*{(e)}
\label{sub:(e)}

Below is the function to display certain data on the LCD:\\

\begin{minted}{c}
  /*
   * @brief Display the specified data on the screen
   * @params displayType: What to display on the screen
   *         value: Data to display for the given type
   * @retval None
   */
  void display(displayType_t displayType, uint32_t value) {
    // Check for what we need to display
    switch (displayType) {
    case TIME:
      if (programState != PROG_STATE_COUNTING) {
        // Only clear the screen if we know that the first line is going to change
        lcd_command(CLEAR);
        lcd_putstring("Time");
      }

      // Convert the time to the string format and display it on the LCD
      lcd_command(LINE_TWO);
      uint8_t *string = time2String(value);
      lcd_putstring(string);
      free(string); // Make sure we de-allocate the string!
      break;
    case WELCOME:
      // Display the welcome message
      lcd_put2String("Stop Watch", "Press SW0...");
      break;
    default:
      break;
    }
  }
\end{minted}
\vspace{0.5cm}

\newpage
\subsection*{(f)}
\label{sub:(f)}

Below is the completed code as tested:\\

\inputminted[firstline=1, lastline=50]{c}{main.c}
\newpage
\inputminted[firstline=51, lastline=100]{c}{main.c}
\newpage
\inputminted[firstline=101, lastline=150]{c}{main.c}
\newpage
\inputminted[firstline=151, lastline=200]{c}{main.c}
\newpage
\inputminted[firstline=201, lastline=250]{c}{main.c}
\newpage
\inputminted[firstline=251, lastline=300]{c}{main.c}
\newpage
\inputminted[firstline=301, lastline=337]{c}{main.c}
\newpage

\end{document}

%----------------------------------------------------------------------------------------
% INSTRUCTIONS AND TEMPLATE COMPONENTS
%----------------------------------------------------------------------------------------

% Set space - \vspace{length}
% Horizontal line - \hrule
% Pagebreak - \clearpage or \newpage - they seem to do the same thing

% Normal equation:
% \begin{equation}
%   p = \frac{p_g - k}{m}
% \end{equation}

% Aligned equation:
% \begin{equation}
%   \begin{split}
%     \rho_5  & = -\frac{\left(1575-1900\right)}{g\left(105\e{-3}-70\e{-3}\right)}\\
%             & = 946.56 \text{ kg/m\(^{-3}\)}\\
%     \rho_6  & = -\frac{\left(1900-2150\right)}{g\left(70\e{-3}-42\e{-3}\right)}\\
%             & = 910.15 \text{ kg/m\(^{-3}\)}\\
%   \end{split}
% \end{equation}

% Table: Can get from the table site!
% \begin{table}[h]
% \centering
% \begin{tabular}{cccc}
% \multicolumn{4}{c}{\cellcolor[HTML]{EFEFEF}DATA MEASUREMENTS} \\ \hline
% \multicolumn{1}{c|}{\begin{tabular}[c]{@{}c@{}}Known Weight \\ Pressure (Psi)\end{tabular}} & \multicolumn{1}{c|}{\begin{tabular}[c]{@{}c@{}}Known Weight\\ Pressure (Bar)\end{tabular}} & \multicolumn{1}{c|}{\begin{tabular}[c]{@{}c@{}}Gauge Pressure\\ (Psi)\end{tabular}} & \begin{tabular}[c]{@{}c@{}}Gauge Pressure\\ (Bar)\end{tabular} \\ \hline
% \multicolumn{1}{c|}{22} & \multicolumn{1}{c|}{1.52} & \multicolumn{1}{c|}{9.43} & 0.65 \\
% \multicolumn{1}{c|}{\textbf{27}} & \multicolumn{1}{c|}{\textbf{1.86}} & \multicolumn{1}{c|}{\textbf{13.78}} & \textbf{0.95} \\
% \multicolumn{1}{c|}{32} & \multicolumn{1}{c|}{2.21} & \multicolumn{1}{c|}{17.40} & 1.20 \\
% \multicolumn{1}{c|}{37} & \multicolumn{1}{c|}{2.55} & \multicolumn{1}{c|}{21.76} & 1.50 \\
% \multicolumn{1}{c|}{42} & \multicolumn{1}{c|}{2.90} & \multicolumn{1}{c|}{26.11} & 1.80 \\
% \multicolumn{1}{c|}{47} & \multicolumn{1}{c|}{3.24} & \multicolumn{1}{c|}{29.01} & 2.00 \\
% \multicolumn{1}{c|}{\textbf{52}} & \multicolumn{1}{c|}{\textbf{3.59}} & \multicolumn{1}{c|}{\textbf{33.36}} & \textbf{2.30} \\
% \multicolumn{1}{c|}{57} & \multicolumn{1}{c|}{3.93} & \multicolumn{1}{c|}{37.71} & 2.60 \\ \hline
% \end{tabular}
% \caption{Dead weight tester and gauge measurements}
% \label{gaugeMeasurements}
% \end{table}

% Picture:
% \begin{figure}[H]
%   \begin{center}
%     \includegraphics[width=0.4\columnwidth]{MEC2022SLab1Exp1}
%     \caption{Apparatus}
%     \label{liquidDiagram}
%   \end{center}
% \end{figure}

% Graph:
% \begin{figure}[H]
%   \centering
%   \begin{tikzpicture}
%     \begin{axis}[
%       xlabel=Known Weight (Psi),
%       ylabel=Reading (Psi),
%       grid=major]
%     \addplot[color=red, smooth] coordinates {
%       (22,9.43)
%       (27,13.78)
%     };
%     \addlegendentry{Gauge}

%     \addplot[color=blue, dashed, smooth] coordinates {
%       (22,22)
%       (27,27)
%     };
%     \addlegendentry{Known weight}

%     \end{axis}
%   \end{tikzpicture}
%   \caption{Dead weight tester and gauge measurements}
%   \label{gaugePlot}
% \end{figure}

% Indented Text: (bit of a hacky way to do it)
% \begin{quote}
% where \(p_g\) is the gauge reading, \(m\) is the error coefficient, \(p\) is the known (``correct'') pressure and \(k\) is the error constant
% \end{quote}

% Code Snippet:
% \begin{minted}{c}
%   void main(void) {
%     init_leds();
%
%     while (TRUE) {
%       __asm("nop");
%     }
%   }
% \end{minted}

% Bibliography:
% \begin{thebibliography}{99}
%   \bibitem{itemname}
%   Author,
%   Date.
%   \emph{Title}.
%   Edition.
%   Press.
%   \url{http://web.iitd.ac.in/~shouri/eel201/tuts/diode_switch.pdf}
% \end{thebibliography}
