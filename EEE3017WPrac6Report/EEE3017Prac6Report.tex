\input{woodamble}                                                       % Custom preamble
%----------------------------------------------------------------------------------------
% NAME AND CLASS SECTION
%----------------------------------------------------------------------------------------

\newcommand{\reportTitle}{Practical 6 Report} % Assignment title
\newcommand{\reportDueDate}{Wednesday,\ September\ 03,\ 2014}                    % Due date
\newcommand{\reportClass}{EEE3017W - Digitals}                               % Course/class
\newcommand{\reportClassTime}{10:30am}                                 % Class/lecture time
\newcommand{\reportClassInstructor}{Jones}                               % Teacher/lecturer
\newcommand{\reportAuthorName}{Sean Wood - WDXSEA003 $|$ Sean Woodgate - WDGSEA001}                           % Your name
\newcommand{\reportDepartment}{Department of Electrical Engineering}           % Department

%----------------------------------------------------------------------------------------
% TITLE PAGE
%----------------------------------------------------------------------------------------

\title{
\begin{figure}[H]
  \begin{center}
    \includegraphics[width=0.4\columnwidth]{uct-logo}
  \end{center}
\end{figure}
\textmd{\Huge UNIVERSITY OF CAPETOWN \\ \LARGE \reportDepartment} \\
\vspace{2in}
\textmd{\textbf{\LARGE \reportClass\\ \Huge \reportTitle \\ \vspace{1.5in} \Large \reportAuthorName}}\\
%\normalsize\vspace{0.1in}\small{Due\ on\ \hmwkDueDate}\\
%\vspace{0.1in}\large{\textit{\hmwkClassInstructor\ \hmwkClassTime}}
}

\date{}

%========================================================================================
%========================================================================================

\begin{document}
  \maketitle
  \thispagestyle{empty}

% \frontmatter
%----------------------------------------------------------------------------------------
% ABTRACT
%----------------------------------------------------------------------------------------
\setcounter{tocdepth}{3}                                                      % ToC Depth

% \begin{abstract}
% This report covers the design and testing of a state feedback controller implemented to control a computer model of a helicopter.  The controller interfaces with the computer by means of a ADC and DAC and is comprised of operational amplifiers. % TODO put more here!s
% \end{abstract}
% \newpage

%----------------------------------------------------------------------------------------
% TABLE OF CONTENTS
%----------------------------------------------------------------------------------------

% \clearpage
%
% \tableofcontents
%
% \clearpage
%
% \listoffigures
% \listoftables
%
% \clearpage
%
%----------------------------------------------------------------------------------------
% NOMENCLATURE
%----------------------------------------------------------------------------------------

% \section{Nomenclature}
% \label{sec:Nomenclature}
%
% \textbf{Constants}
%
% \(V_{LINE}\) - Nominal Line Voltage\\
% \(A_v\) - Operational Amplifier Gain\\
% \(db_{SPL}\) - Sound Pressure Level\\
% \(\theta_S\) - Phase Shift

%----------------------------------------------------------------------------------------
% INTRODUCTION
%----------------------------------------------------------------------------------------
\section{Introduction}
\label{sec:Introduction}
This report covers the system design and implementation of a rain-gauge on the STM32F051C6 Development Board as part of the EEE3017W Digitals Practical 6.

\section*{System Design}
\label{sec:System Design}
\subsection*{(a)}
\label{sub:(a)}
Below is the initialization function for the ports used in the project: \\
\begin{mdframed}[linecolor=black, topline=false, bottomline=false,
  leftline=false, rightline=false, userdefinedwidth=\textwidth]
  \begin{minted}{c}
    /*
    * @brief Initialise the GPIO ports for pushbuttons, LEDs and the ADC
    * @params None
    * @retval None
    */
    static void init_ports(void) {
      // Enable the clock for ports used
      RCC->AHBENR |= RCC_AHBENR_GPIOBEN | RCC_AHBENR_GPIOAEN;

      // Initialise PB0 - PB7, PB10 and PB11 for RG Led
      GPIOB->MODER |= GPIO_MODER_MODER0_0 | GPIO_MODER_MODER1_0 |
                      GPIO_MODER_MODER2_0 | GPIO_MODER_MODER3_0 |
                      GPIO_MODER_MODER4_0 | GPIO_MODER_MODER5_0 |
                      GPIO_MODER_MODER6_0 | GPIO_MODER_MODER7_0 |
                      GPIO_MODER_MODER10_0 | GPIO_MODER_MODER11_0;
      GPIOB->ODR &= ~(GPIO_ODR_10 | GPIO_ODR_11); // Make sure they are not on

      // Initialise PA0, PA1, PA2 and PA3 for SW0, SW1, SW2 and SW3
      GPIOA->MODER &= ~(GPIO_MODER_MODER0 | GPIO_MODER_MODER1 | GPIO_MODER_MODER2
          | GPIO_MODER_MODER3);
      GPIOA->PUPDR |= GPIO_PUPDR_PUPDR0_0 | GPIO_PUPDR_PUPDR1_0
          | GPIO_PUPDR_PUPDR2_0 | GPIO_PUPDR_PUPDR3_0; // Enable pullup resistors

      // Initialise PA5 for ADC1
      GPIOA->MODER |= GPIO_MODER_MODER5;
    }
  \end{minted}
\end{mdframed}
\vspace{0.5cm}

Below is the initialization function for the Analog to Digital Converter (ADC):\\

\begin{minted}{c}
  /*
   * @brief Initialise the ADC to POT0
   * @params None
   * @retval None
   */
  static void init_ADC(void) {
    // Enable the ADC clock in the RCC
    RCC->APB2ENR |= RCC_APB2ENR_ADCEN;

    // Select ADC channel 5 for POT0
    ADC1->CHSELR |= ADC_CHSELR_CHSEL5;

    // Enable the ADC peripheral
    ADC1->CR |= ADC_CR_ADEN;

    // Wait for the ADC to become ready
    while (!(ADC1->ISR & ADC_ISR_ADRDY)) {
      __asm("nop");
    }
  }
\end{minted}
\vspace{0.5cm}

Below is the initialization function for the Nested Vector Interrupt Controller (NVIC):\\

\begin{minted}{c}
  /*
   * @brief Initialise the NVIC for pushbutton interrupts
   * @params None
   * @retval None
   */
  static void init_NVIC(void) {
    NVIC_EnableIRQ(EXTI0_1_IRQn); // For lines 0 and 1
    NVIC_EnableIRQ(EXTI2_3_IRQn); // For lines 2 and 3
    NVIC_EnableIRQ(TIM14_IRQn); // For TIM14
  }
\end{minted}
\vspace{0.5cm}

Below is the initialization function for the External Interrupt controller (EXTI):\\

\begin{minted}{c}
  /*
   * @brief Initialise the EXTI lines for pushbutton interrupts
   * @params None
   * @retval None
   */
  static void init_EXTI(void) {
    RCC->APB2ENR |= RCC_APB2ENR_SYSCFGCOMPEN; // Enable the SYSCFG and COMP RCC clock
    SYSCFG->EXTICR[1] &= ~(0xFFFF); // Map PA0 and PA1 to external interrupt lines

    EXTI->FTSR |= EXTI_FTSR_TR0 | EXTI_FTSR_TR1 | EXTI_FTSR_TR2 | EXTI_FTSR_TR3; // Configure trigger to falling edge
    EXTI->IMR |= EXTI_IMR_MR0 | EXTI_IMR_MR1 | EXTI_IMR_MR2 | EXTI_IMR_MR3; // Umask the interrupts
  }
\end{minted}
\vspace{0.5cm}

\subsection*{(b)}
\label{sub:(b)}
The rain-gauge requires a low voltage detection system to detect when the battery falls below a certain threshold. Given that the maximum battery voltage is 24V and the threshold is set to 14V, the digital ADC value at the threshold will be the following:

\begin{equation*}
  \begin{split}
    \text{ADC Analog Value} \biggr\rvert_{\text{at threshold}} &= \frac{V_{ADC_{max}}}{V_{bat_{max}}} \times V_{thres}\\
    &= \frac{3}{24} (14)\\
    &= 1.75 \text{V}
  \end{split}
\end{equation*}

\begin{equation*}
  \begin{split}
    \text{ADC}\biggr\rvert_{\text{1.75 V}} &= \frac{\text{12-bit maximim}}{V_{ADC_{max}}} \times \text{ADC Analog Value}\\
    &= \frac{4095}{3} (1.75)\\
    &= 2399 \eoc
  \end{split}
\end{equation*}

\subsection*{(c)}
\label{sub:(c)}

Below are the functions to monitor the battery voltage:\\

\begin{minted}{c}
    /*
   * @brief Kick off and grab an ADC conversion
   * @params None
   * @retval None
   */
  static uint16_t getADC(void) {
    // Start a conversion
    ADC1->CR |= ADC_CR_ADSTART;

    // Wait for the conversion to finish
    while (!(ADC1->ISR & ADC_ISR_EOC)) {
      __asm("nop");
    }

    // Return the result of the conversion
    return (uint16_t)(ADC1->DR);
  }

  /*
   * @brief Check the "battery voltage" and display it
   * @params None
   * @retval None
   */
  static void check_battery(void) {
    // Grab the ADC value, convert to uV and then to battery voltage
    uint16_t adcVal = getADC();
    uint32_t uVoltage = adcVal * ADC_GRAIN;
    batVoltage = 7.21*(uVoltage/ADC_MULTIPLIER);

    // Check for voltage threshold and change the LED accordingly
    if (batVoltage <= BAT_THRESHOLD) {
      GPIOB->ODR &= ~(1 << 11);
      GPIOB->ODR |= (1 << 10);
    } else {
      GPIOB->ODR &= ~(1 << 10);
      GPIOB->ODR |= (1 << 11);
    }
  }
\end{minted}
\vspace{0.5cm}

\subsection*{(d)}
\label{sub:(d)}

Below is the EXTI handler for lines 0 and 1:\\

\begin{minted}{c}
  /*
   * @brief Interrupt Request Handler for EXTI Lines 2 and 3 (PB0 and PB1)
   * @params None
   * @retval None
   */
  void EXTI0_1_IRQHandler(void) {
    // Check which button generated the interrupt
    if (getSW(0)) {
      // Check the state of the program
      switch (programState) {
      case PROG_STATE_WAIT_FOR_SW0:
        // If we were waiting for SW0, display the menu
        display(DISP_MENU, 0);

        // Change program state
        programState = PROG_STATE_WAIT_FOR_BUTTON;
        break;
      default:
        break;
      }
    } else if (getSW(1)) {
      // Check the state of the program
      switch (programState) {
      case PROG_STATE_WAIT_FOR_BUTTON:
        // If we were waiting for another button:
        rainCounter++; // Increment the rain counter
        display(DISP_RAIN_BUCKET, 0); // Notify the user
        break;
      default:
        break;
      }
    }

    // Clear the interrupt pending bit
    EXTI->PR |= EXTI_PR_PR0 | EXTI_PR_PR1;
  }
\end{minted}
\vspace{0.5cm}

\subsection*{(e)}
\label{sub:(e)}

Below is the function to convert floats to BCD:\\

\begin{minted}{c}
  /*
   * @brief Convert the float given to a string
   * @params rain: Rain in mm
   *         dec: Number of digits to the left of the decimal point
   *         frac: Number of decimal places (precision)
   * @retval Pointer to the converted string
   * @note String must be freed after use
   */
  static uint8_t *ConverttoBCD(float number, uint8_t dec, uint8_t frac) {
    uint8_t *string; // Pointer to the resulting string
    uint32_t rainDec = number*pow(10,frac); // Shift all digits to be used onto the left side of the decimal point
    uint32_t strLength = (dec + frac + 2)*sizeof(uint8_t); // Calculate the length of the require string given the accuracy parameters
    string = malloc(strLength); // Allocate space for the resulting string
    memset(string, '0', strLength); // Set all characters in the string to zeroes

    // Loop through the digits in the newly formed integer number and place the digits in the string
    int pos = 0;
    int dig = 0;
    for (pos = 0; pos < strLength; pos++) {
      // If we reach the end of the decimal part of the number, skip a position for placement of the decimal point
      if (pos == dec) {
        pos++;
      }

      // Extract the digit from the newly formed integer number based on the position
      uint32_t multiplier = pow(10, strLength-dig-3);
      uint32_t digit = (uint32_t)(rainDec/multiplier);
      string[pos] = (uint8_t)(digit + 48); // Convert the number to ASCII by adding 48 to it
      rainDec -= digit*multiplier; // Subtract the extracted digit from the integer number

      // Increment the digit number
      dig++;
    }

    // Place the decimal point and the null terminator in the correct positions
    string[dec] = '.';
    string[strLength - 1] = '\0';

    // Return the pointer to the converted string
    return string;
  }
\end{minted}
\vspace{0.5cm}

\subsection*{(e)}
\label{sub:(e)}

Below is the function to display values on the LCD:\\

\begin{minted}{c}
  /*
   * @brief Display the specified data on the screen
   * @params displayType: What to display on the screen
   *         ...: Data to display for the given type
   * @retval None
   */
  void display(displayType_t displayType, float data) {
    // Switch on what needs to be displayed
    switch (displayType) {
    case DISP_BAT: {
      // Display the battery voltage on the LCD
      lcd_command(CLEAR);
      lcd_command(CURSOR_HOME);
      lcd_putstring("Battery:");
      lcd_command(LINE_TWO);

      // Generate the string with the batter voltage
      uint8_t *string = ConverttoBCD(data, 2, 3);
      lcd_putstring(string);
      lcd_putstring(" V");

      // De-allocate the memory used for the battery string
      free(string);
      break;
    }
    case DISP_RAINFALL: {
      // Display the rainfall amount on the LCD
      lcd_command(CLEAR);
      lcd_command(CURSOR_HOME);
      lcd_putstring("Rainfall:");
      lcd_command(LINE_TWO);

      // Fetch and convert the rainfall to a string
      float rain = 0.2*data;
      uint8_t *string = ConverttoBCD(rain, 4, 1);
      lcd_putstring(string);
      lcd_putstring(" mm");

      // De-allocate the memory used for the rainfall string
      free(string);
      break;
    }
    case DISP_RAIN_BUCKET:
      // Display the bucket tip notification LCD
      lcd_put2String("Rain bucket tip", "");
      break;
    case DISP_MENU:
      // Display the menu on the LCD
      lcd_put2String("Weather Station", "Press SW2 or SW3");
      break;
    case DISP_WELCOME:
      // Display the welcome on the LCD
      lcd_put2String("EEE3017W Prac 6", "Sean & Sean");
      break;
    default:
      break;
    }
  }
\end{minted}
\vspace{0.5cm}

\subsection*{(f)}
\label{sub:(f)}

Below is the completed code as tested:\\

\inputminted[firstline=1, lastline=50]{c}{main.c}
\newpage
\inputminted[firstline=51, lastline=100]{c}{main.c}
\newpage
\inputminted[firstline=101, lastline=150]{c}{main.c}
\newpage
\inputminted[firstline=151, lastline=200]{c}{main.c}
\newpage
\inputminted[firstline=201, lastline=250]{c}{main.c}
\newpage
\inputminted[firstline=251, lastline=300]{c}{main.c}
\newpage
\inputminted[firstline=301, lastline=350]{c}{main.c}
\newpage
\inputminted[firstline=351, lastline=400]{c}{main.c}
\newpage
\inputminted[firstline=401, lastline=450]{c}{main.c}

\vspace{0.5cm}


\end{document}

%----------------------------------------------------------------------------------------
% INSTRUCTIONS AND TEMPLATE COMPONENTS
%----------------------------------------------------------------------------------------

% Set space - \vspace{length}
% Horizontal line - \hrule
% Pagebreak - \clearpage or \newpage - they seem to do the same thing

% Normal equation:
% \begin{equation}
%   p = \frac{p_g - k}{m}
% \end{equation}

% Aligned equation:
% \begin{equation}
%   \begin{split}
%     \rho_5  & = -\frac{\left(1575-1900\right)}{g\left(105\e{-3}-70\e{-3}\right)}\\
%             & = 946.56 \text{ kg/m\(^{-3}\)}\\
%     \rho_6  & = -\frac{\left(1900-2150\right)}{g\left(70\e{-3}-42\e{-3}\right)}\\
%             & = 910.15 \text{ kg/m\(^{-3}\)}\\
%   \end{split}
% \end{equation}

% Table: Can get from the table site!
% \begin{table}[h]
% \centering
% \begin{tabular}{cccc}
% \multicolumn{4}{c}{\cellcolor[HTML]{EFEFEF}DATA MEASUREMENTS} \\ \hline
% \multicolumn{1}{c|}{\begin{tabular}[c]{@{}c@{}}Known Weight \\ Pressure (Psi)\end{tabular}} & \multicolumn{1}{c|}{\begin{tabular}[c]{@{}c@{}}Known Weight\\ Pressure (Bar)\end{tabular}} & \multicolumn{1}{c|}{\begin{tabular}[c]{@{}c@{}}Gauge Pressure\\ (Psi)\end{tabular}} & \begin{tabular}[c]{@{}c@{}}Gauge Pressure\\ (Bar)\end{tabular} \\ \hline
% \multicolumn{1}{c|}{22} & \multicolumn{1}{c|}{1.52} & \multicolumn{1}{c|}{9.43} & 0.65 \\
% \multicolumn{1}{c|}{\textbf{27}} & \multicolumn{1}{c|}{\textbf{1.86}} & \multicolumn{1}{c|}{\textbf{13.78}} & \textbf{0.95} \\
% \multicolumn{1}{c|}{32} & \multicolumn{1}{c|}{2.21} & \multicolumn{1}{c|}{17.40} & 1.20 \\
% \multicolumn{1}{c|}{37} & \multicolumn{1}{c|}{2.55} & \multicolumn{1}{c|}{21.76} & 1.50 \\
% \multicolumn{1}{c|}{42} & \multicolumn{1}{c|}{2.90} & \multicolumn{1}{c|}{26.11} & 1.80 \\
% \multicolumn{1}{c|}{47} & \multicolumn{1}{c|}{3.24} & \multicolumn{1}{c|}{29.01} & 2.00 \\
% \multicolumn{1}{c|}{\textbf{52}} & \multicolumn{1}{c|}{\textbf{3.59}} & \multicolumn{1}{c|}{\textbf{33.36}} & \textbf{2.30} \\
% \multicolumn{1}{c|}{57} & \multicolumn{1}{c|}{3.93} & \multicolumn{1}{c|}{37.71} & 2.60 \\ \hline
% \end{tabular}
% \caption{Dead weight tester and gauge measurements}
% \label{gaugeMeasurements}
% \end{table}

% Picture:
% \begin{figure}[H]
%   \begin{center}
%     \includegraphics[width=0.4\columnwidth]{MEC2022SLab1Exp1}
%     \caption{Apparatus}
%     \label{liquidDiagram}
%   \end{center}
% \end{figure}

% Graph:
% \begin{figure}[H]
%   \centering
%   \begin{tikzpicture}
%     \begin{axis}[
%       xlabel=Known Weight (Psi),
%       ylabel=Reading (Psi),
%       grid=major]
%     \addplot[color=red, smooth] coordinates {
%       (22,9.43)
%       (27,13.78)
%     };
%     \addlegendentry{Gauge}

%     \addplot[color=blue, dashed, smooth] coordinates {
%       (22,22)
%       (27,27)
%     };
%     \addlegendentry{Known weight}

%     \end{axis}
%   \end{tikzpicture}
%   \caption{Dead weight tester and gauge measurements}
%   \label{gaugePlot}
% \end{figure}

% Indented Text: (bit of a hacky way to do it)
% \begin{quote}
% where \(p_g\) is the gauge reading, \(m\) is the error coefficient, \(p\) is the known (``correct'') pressure and \(k\) is the error constant
% \end{quote}

% Code Snippet:
% \begin{minted}{c}
%   void main(void) {
%     init_leds();
%
%     while (TRUE) {
%       __asm("nop");
%     }
%   }
% \end{minted}

% Bibliography:
% \begin{thebibliography}{99}
%   \bibitem{itemname}
%   Author,
%   Date.
%   \emph{Title}.
%   Edition.
%   Press.
%   \url{http://web.iitd.ac.in/~shouri/eel201/tuts/diode_switch.pdf}
% \end{thebibliography}
